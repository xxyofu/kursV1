\section{Техническое задание}
\subsection{Основание для разработки}

Основанием для разработки является задание на курсовую работу "<Разработка библиотеки для чтения изображений формата JPEG">.

\subsection{Цель и назначение разработки}

Основной задачей курсовой работы является разработка библиотеки для чтения и декодирования JPEG изображений.

Задачами данной разработки являются:
\begin{itemize}
\item Создание функции для чтения файла формата JPEG;
\item Создание функции для декодирования маркеров файла JPEG;
\item Создание функций для декодирования сжатого набора данных JPEG файла.
\end{itemize}

\subsection{Требования пользователя к интерфейсу web-сайта}

Т.к. создаваемый проект является библиотекой, то пользовательский интерфейс будет программным, то есть реализован через прямое взаимодействие пользователя с функциями библиотеки.

Библиотека должена включать в себя следующие функции:
\begin{itemize}
    \item Функция для чтения данных изображения;
    \item Функция определения метаданных изображения по маркерам;
    \item Функция нахождения DC, AC коэффициентов;
    \item Функция обратная квантованию;
    \item Функция обратного дискретно-косинусного преобразования;
    \item Функция преобразования изображения в другую цветовую палитру.
\end{itemize}

\subsection{Моделирование вариантов использования}

Для разрабатываемого сайта была реализована модель, которая обеспечивает наглядное представление вариантов использования библиотеки.

Она помогает в физической разработке и детальном анализе взаимосвязей объектов. При построении диаграммы вариантов использования применяется унифицированный язык визуального моделирования UML.

Диаграмма вариантов описывает функциональное назначение разрабатываемой системы. То есть это то, что система будет непосредственно делать в процессе своего функционирования. Она является исходным концептуальным представлением системы в процессе ее проектирования и разработки. Проектируемая система представляется в виде ряда прецедентов, предоставляемых системой актерам или сущностям, которые взаимодействуют с системой. Актером или действующим лицом является сущность, взаимодействующая с системой извне (например, человек, техническое устройство). Прецедент служит для описания набора действий, которые система предоставляет актеру.

На основании анализа предметной области в программе должны быть реализованы следующие прецеденты:
\begin{enumerate}
\item Открытие файла формата JPEG.
\item Чтение файла формата JPEG.
\item Декодирование файла формата JPEG.
\end{enumerate}

\subsection{Требования к оформлению документации}

Разработка программной документации и программного изделия должна производиться согласно ГОСТ 19.102-77 и ГОСТ 34.601-90. Единая система программной документации.
