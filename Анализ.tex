\section{Анализ предметной области}
\subsection{Введение и актуальность}

В современном цифровом мире растровые изображения являются фундаментальным типом данных. Они повсеместно используются в веб-разработке, мобильных приложениях, научных исследованиях (медицина, астрономия), машинном обучении (компьютерное зрение), дизайне и фотографии.

Обработка изображений — ресурсоемкая задача, требующая эффективных и специализированных инструментов. Существует множество библиотек (OpenCV, Pillow, ImageMagick, libpng, libjpeg-turbo), однако их зависимости, сложность интеграции или лицензионные ограничения часто создают препятствия для разработчиков.

\subsection{Типы растровых изображений}

Библиотеки обрабатывают растровые изображения, читают и записывает их, для работы с ними необходимо знать каких типов бывают растровые изображения и в чем их разница. Основное отличие типов растровых изображений в их хранении и записи, некоторые форматы сжимают данные и или удаляют ненужные куски данных, в зависимости от предназначения, выбираются различные форматы изображений.

Наиболее популярные из них:
\begin{enumerate}
\item PNG, этот формат использует алгоритм сжатия без потерь(LZ77 - Deflate), в особенности этого формата входит поддержка алфа-канала
\item JPEG, этот формат исползует алгоритм сжатия с потерями(Дискретное косинусное преобразование)
\item PPM, особенность этого формата в том, что он не использует алгоритмы сжатия, и хранит данные, как они есть, каждый пиксель в виде числа
\end{enumerate}

\subsection{Библиотеки для работы с изображениями}

Уже существует огромное множество библиотек предназначенных для различных языков и различных задач для работы с изображениями. Они отличаются открытостью кода, удобством разным подходом к представлению данных. Каждый выбирает одну из них на основе поставленной задачи, например, для работы с компьютерным зрением подойдет OpenCV, а для простой обработки изображений на Python хватит и Pillow.

Основные библиотеки и их особенности:
\begin{enumerate}
\item OpenCV, мощная библиотека для обработки изображений, использующаяся для работы с компьютерным зрением, существует почти для любого языка программирования, но очень большая по размеру
\item Pillow, библиотека для работы с изображениями, их чтением, обработкой, записью, но существует только для языка Python
\item ImageMagick, библиотека с открытым исходным кодом, включает в себя широкий набор функций, подходит для работы с широким списком форматов
\end{enumerate}

Несмотря на различия в функциональности этих библиотек, они имеют ряд общих черт, которые должны быть у любой библиотеки для работы с изображениями.
Основные функции:
\begin{enumerate}
\item Поддержка различных форматов изображений(PNG, JPEG, TIFF и др.)
\item Чтение и запись изображений
\item Возможность обрабатывать или редактировать изображение
\end{enumerate}

