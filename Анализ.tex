\section{Анализ предметной области}
\subsection{Введение и актуальность}

В современном цифровом мире растровые изображения являются фундаментальным типом данных. Они повсеместно используются в веб-разработке, мобильных приложениях, научных исследованиях (медицина, астрономия), машинном обучении (компьютерное зрение), дизайне и фотографии.

Обработка изображений — ресурсоемкая задача, требующая эффективных и специализированных инструментов. Существует множество библиотек (OpenCV, Pillow, ImageMagick, libpng, libjpeg-turbo), однако их зависимости, сложность интеграции или лицензионные ограничения часто создают препятствия для разработчиков.

\subsection{Типы растровых изображений}

Библиотеки обрабатывают растровые изображения, читают и записывает их, для работы с ними необходимо знать каких типов бывают растровые изображения и в чем их разница. Основное отличие типов растровых изображений в их хранении и записи, некоторые форматы сжимают данные и или удаляют ненужные куски данных, в зависимости от предназначения, выбираются различные форматы изображений.

Наиболее популярные из них:
\begin{enumerate}
\item PNG, этот формат использует алгоритм сжатия без потерь(LZ77 - Deflate), в особенности этого формата входит поддержка алфа-канала
\item JPEG, этот формат исползует алгоритм сжатия с потерями(Дискретное косинусное преобразование)
\item PPM, особенность этого формата в том, что он не использует алгоритмы сжатия, и хранит данные, как они есть, каждый пиксель в виде числа
\end{enumerate}

\subsection{Библиотеки для работы с изображениями}

Уже существует огромное множество библиотек предназначенных для различных языков и различных задач для работы с изображениями. Они отличаются открытостью кода, удобством разным подходом к представлению данных. Каждый выбирает одну из них на основе поставленной задачи, например, для работы с компьютерным зрением подойдет OpenCV, а для простой обработки изображений на Python хватит и Pillow.

Основные библиотеки и их особенности:
\begin{enumerate}
\item OpenCV, мощная библиотека для обработки изображений, использующаяся для работы с компьютерным зрением, существует почти для любого языка программирования, но очень большая по размеру
\item Pillow, библиотека для работы с изображениями, их чтением, обработкой, записью, но существует только для языка Python
\item ImageMagick, библиотека с открытым исходным кодом, включает в себя широкий набор функций, подходит для работы с широким списком форматов
\end{enumerate}

Несмотря на различия в функциональности этих библиотек, они имеют ряд общих черт, которые должны быть у любой библиотеки для работы с изображениями.
Основные функции:
\begin{enumerate}
\item Поддержка различных форматов изображений(PNG, JPEG, TIFF и др.)
\item Чтение и запись изображений
\item Возможность обрабатывать или редактировать изображение
\end{enumerate}

\subsection{Формат JPEG}

JPEG — один из популярных растровых графических форматов, применяемый для хранения фотографий и подобных им изображений. Файлы, содержащие данные JPEG, обычно имеют расширения (суффиксы) .jpg (самое популярное), .jfif, .jpe или .jpeg. MIME-тип — image/jpeg.

Алгоритм JPEG позволяет сжимать изображение как с потерями, так и без потерь (режим сжатия lossless JPEG). Поддерживаются изображения с линейным размером не более 65535 × 65535 пикселов.

Процесс сжатия JPEG изображения происходит в несколько этапов, сначала изображение преобразуется из RGB цветового пространства в YCbCr, после этого оно проходит дискретное косинусное преобразование, а в конце кодируются длины серий, делты, Хаффмана.

Стандарт JPEG предусматривает два основных способа представления кодируемых данных.

Наиболее распространённым, поддерживаемым большинством доступных кодеков, является последовательное (sequential JPEG) представление данных, предполагающее последовательный обход кодируемого изображения разрядностью 8 бит на компоненту (или 8 бит на пиксель для чёрно-белых полутоновых изображений) поблочно слева направо, сверху вниз. Над каждым кодируемым блоком изображения осуществляются описанные выше операции, а результаты кодирования помещаются в выходной поток в виде единственного «скана», то есть массива кодированных данных, соответствующего последовательно пройденному («просканированному») изображению. Основной или «базовый» (baseline) режим кодирования допускает только такое представление (и хаффмановское кодирование квантованных коэффициентов ДКП). Расширенный (extended) режим наряду с последовательным допускает также прогрессивное (progressive JPEG) представление данных, кодирование изображений разрядностью 12 бит на компоненту/пиксель (сжатие таких изображений спецификацией JFIF не поддерживается) и арифметическое кодирование квантованных коэффициентов ДКП.

В случае progressive JPEG сжатые данные записываются в выходной поток в виде набора сканов, каждый из которых описывает изображение полностью с всё большей степенью детализации. Это достигается либо путём записи в каждый скан не полного набора коэффициентов ДКП, а лишь какой-то их части: сначала — низкочастотных, в следующих сканах — высокочастотных (метод «spectral selection» то есть спектральных выборок), либо путём последовательного, от скана к скану, уточнения коэффициентов ДКП (метод «successive approximation», то есть последовательных приближений). Такое прогрессивное представление данных оказывается особенно полезным при передаче сжатых изображений с использованием низкоскоростных каналов связи, поскольку позволяет получить представление обо всём изображении уже после передачи незначительной части JPEG-файла.

Алгоритм JPEG наиболее эффективен для сжатия фотографий и картин, содержащих реалистичные сцены с плавными переходами яркости и цвета. Наибольшее распространение JPEG получил в цифровой фотографии и для хранения и передачи изображений с использованием Интернета.
Формат JPEG в режиме сжатия с потерями малопригоден для сжатия чертежей, текстовой и знаковой графики, где резкий контраст между соседними пикселами приводит к появлению заметных артефактов. Такие изображения целесообразно сохранять в форматах без потерь, таких как JPEG-LS, TIFF, GIF, PNG, либо использовать режим сжатия Lossless JPEG.
JPEG (как и другие форматы сжатия с потерями) не подходит для сжатия изображений при многоэтапной обработке, так как искажения в изображения будут вноситься каждый раз при сохранении промежуточных результатов обработки.
JPEG не должен использоваться и в тех случаях, когда недопустимы даже минимальные потери, например при сжатии астрономических или медицинских изображений. В таких случаях может быть рекомендован предусмотренный стандартом JPEG режим сжатия Lossless JPEG (который, однако, не поддерживается большинством популярных кодеков) или стандарт сжатия JPEG-LS.
